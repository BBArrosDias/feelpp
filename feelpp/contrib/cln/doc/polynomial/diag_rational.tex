%% This LaTeX-file was created by <bruno> Sun Feb 16 14:19:08 1997
%% LyX 0.10 (C) 1995 1996 by Matthias Ettrich and the LyX Team

%% Don't edit this file unless you are sure what you are doing.
\documentclass[12pt,a4paper,oneside,onecolumn]{article}
\usepackage[]{fontenc}
\usepackage[latin1]{inputenc}
\usepackage[dvips]{epsfig}

%%
%% BEGIN The lyx specific LaTeX commands.
%%

\makeatletter
\def\LyX{L\kern-.1667em\lower.25em\hbox{Y}\kern-.125emX\spacefactor1000}%
\newcommand{\lyxtitle}[1] {\thispagestyle{empty}
\global\@topnum\z@
\section*{\LARGE \centering \sffamily \bfseries \protect#1 }
}
\newcommand{\lyxline}[1]{
{#1 \vspace{1ex} \hrule width \columnwidth \vspace{1ex}}
}
\newenvironment{lyxsectionbibliography}
{
\section*{\refname}
\@mkboth{\uppercase{\refname}}{\uppercase{\refname}}
\begin{list}{}{
\itemindent-\leftmargin
\labelsep 0pt
\renewcommand{\makelabel}{}
}
}
{\end{list}}
\newenvironment{lyxchapterbibliography}
{
\chapter*{\bibname}
\@mkboth{\uppercase{\bibname}}{\uppercase{\bibname}}
\begin{list}{}{
\itemindent-\leftmargin
\labelsep 0pt
\renewcommand{\makelabel}{}
}
}
{\end{list}}
\def\lxq{"}
\newenvironment{lyxcode}
{\list{}{
\rightmargin\leftmargin
\raggedright
\itemsep 0pt
\parsep 0pt
\ttfamily
}%
\item[]
}
{\endlist}
\newcommand{\lyxlabel}[1]{#1 \hfill}
\newenvironment{lyxlist}[1]
{\begin{list}{}
{\settowidth{\labelwidth}{#1}
\setlength{\leftmargin}{\labelwidth}
\addtolength{\leftmargin}{\labelsep}
\renewcommand{\makelabel}{\lyxlabel}}}
{\end{list}}
\newcommand{\lyxletterstyle}{
\setlength\parskip{0.7em}
\setlength\parindent{0pt}
}
\newcommand{\lyxaddress}[1]{
\par {\raggedright #1 
\vspace{1.4em}
\noindent\par}
}
\newcommand{\lyxrightaddress}[1]{
\par {\raggedleft \begin{tabular}{l}\ignorespaces
#1
\end{tabular}
\vspace{1.4em}
\par}
}
\newcommand{\lyxformula}[1]{
\begin{eqnarray*}
#1
\end{eqnarray*}
}
\newcommand{\lyxnumberedformula}[1]{
\begin{eqnarray}
#1
\end{eqnarray}
}
\makeatother

%%
%% END The lyx specific LaTeX commands.
%%

\pagestyle{plain}
\setcounter{secnumdepth}{3}
\setcounter{tocdepth}{3}

%% Begin LyX user specified preamble:
\catcode`@=11 % @ ist ab jetzt ein gewoehnlicher Buchstabe
\def\Res{\mathop{\operator@font Res}}
\def\ll{\langle\!\langle}
\def\gg{\rangle\!\rangle}
\catcode`@=12 % @ ist ab jetzt wieder ein Sonderzeichen


%% End LyX user specified preamble.
\begin{document}


\title{The diagonal of a rational function}

\begin{description}

\item [Theorem:]~

\end{description}

Let  \( M \) be a torsion-free  \( R \)-module, and  \( d>0 \). Let 
\[
f=\sum _{n_{1},...,n_{d}}a_{n_{1},...,n_{d}}\, x_{1}^{n_{1}}\cdots x_{d}^{n_{d}}\in M[[x_{1},\ldots x_{d}]]\]
be a rational function,
i.e. there are  \( P\in M[x_{1},\ldots ,x_{d}] \) and  \( Q\in R[x_{1},\ldots ,x_{d}] \) with  \( Q(0,\ldots ,0)=1 \) and  \( Q\cdot f=P \). Then the full diagonal of  \( f \)
\[
g=\sum ^{\infty }_{n=0}a_{n,\ldots ,n}\, x_{1}^{n}\]
is
a D-finite element of  \( M[[x_{1}]] \), w.r.t.  \( R[x_{1}] \) and  \( \{\partial _{x_{1}}\} \).

\begin{description}

\item [Proof:]~

\end{description}

From the hypotheses,  \( M[[x_{1},\ldots ,x_{d}]] \) is a torsion-free differential module over
 \( R[x_{1},\ldots ,x_{d}] \) w.r.t. the derivatives  \( \{\partial _{x_{1}},\ldots ,\partial _{x_{d}}\} \), and  \( f \) is a D-finite element of  \( M[[x_{1},\ldots ,x_{d}]] \) over
 \( R[x_{1},\ldots ,x_{d}] \) w.r.t.  \( \{\partial _{x_{1}},\ldots ,\partial _{x_{d}}\} \). Now apply the general diagonal theorem ([1], section 2.18)
 \( d-1 \) times.

\begin{description}

\item [Theorem:]~

\end{description}

Let  \( R \) be an integral domain of characteristic 0 and  \( M \) simultaneously
a torsion-free  \( R \)-module and a commutative  \( R \)-algebra without zero divisors.
Let 
\[
f=\sum _{m,n\geq 0}a_{m,n}x^{m}y^{n}\in M[[x,y]]\]
 be a rational function. Then the diagonal of  \( f \)
\[
g=\sum ^{\infty }_{n=0}a_{n,n}\, x^{n}\]
 is algebraic
over  \( R[x] \).

\begin{description}

\item [Motivation~of~proof:]~

\end{description}

The usual proof ([2]) uses complex analysis and works only for  \( R=M=C \).
The idea is to compute
\[
g(x^{2})=\frac{1}{2\pi i}\oint _{|z|=1}f(xz,\frac{x}{z})\frac{dz}{z}\]
This integral, whose integrand is a rational
function in  \( x \) and  \( z \), is calculated using the residue theorem. Since
 \( f(x,y) \) is continuous at  \( (0,0) \), there is a  \( \delta >0 \) such that  \( f(x,y)\neq \infty  \) for  \( |x|<\delta  \),  \( |y|<\delta  \). It follows
that for all  \( \varepsilon >0 \) and  \( |x|<\delta \varepsilon  \) all the poles of  \( f(xz,\frac{x}{z}) \) are contained in  \( \{z:|z|<\varepsilon \}\cup \{z:|z|>\frac{1}{\varepsilon }\} \). Thus the
poles of  \( f(xz,\frac{x}{z}) \), all algebraic functions of  \( x \) -- let's call them  \( \zeta _{1}(x),\ldots \zeta _{s}(x) \) --,
can be divided up into those for which  \( |\zeta _{i}(x)|=O(|x|) \) as  \( x\rightarrow 0 \) and those for which
 \( \frac{1}{|\zeta _{i}(x)|}=O(|x|) \) as  \( x\rightarrow 0 \). It follows from the residue theorem that for  \( |x|<\delta  \)
\[
g(x^{2})=\sum _{\zeta =0\vee \zeta =O(|x|)}\Res _{z=\zeta }\, f(xz,\frac{x}{z})\]
 This is algebraic
over  \( C(x) \). Hence  \( g(x) \) is algebraic over  \( C(x^{1/2}) \), hence also algebraic over  \( C(x) \).

\begin{description}

\item [Proof:]~

\end{description}

Let 
\[
h(x,z):=f(xz,\frac{x}{z})=\sum ^{\infty }_{m,n=0}a_{m,n}x^{m+n}z^{m-n}\in M[[xz,xz^{-1}]]\]
Then  \( g(x^{2}) \) is the coefficient of  \( z^{0} \) in  \( h(x,z) \). Let  \( N(x,z):=z^{d}Q(xz,\frac{x}{z}) \) (with  \( d:=\max (\deg _{y}P,\deg _{y}Q) \)) be ``the denominator''
of  \( h(x,z) \). We have  \( N(x,z)\in R[x,z] \) and  \( N\neq 0 \) (because  \( N(0,z)=z^{d} \)). Let  \( K \) be the quotient field of
 \( R \). Thus  \( N(x,z)\in K[x][z]\setminus \{0\} \).

It is well-known (see [3], p.64, or [4], chap. IV, �2, prop. 8, or
[5], chap. III, �1) that the splitting field of  \( N(x,z) \) over  \( K(x) \) can be embedded
into a field  \( L((x^{1/r})) \), where  \( r \) is a positive integer and  \( L \) is a finite-algebraic
extension field of  \( K \), i.e. a simple algebraic extension  \( L=K(\alpha )=K\alpha ^{0}+\cdots +K\alpha ^{u-1} \). 

 \( \widetilde{M}:=(R\setminus \{0\})^{-1}\cdot M \) is a  \( K \)-vector space and a commutative  \( K \)-algebra without zero divisors.
 \( \widehat{M}:=\widetilde{M}\alpha ^{0}+\cdots +\widetilde{M}\alpha ^{u-1} \) is an  \( L \)-vector space and a commutative  \( L \)-algebra without zero divisors.



\begin{eqnarray*}
\widehat{M}\ll x,z\gg  & := & \widehat{M}[[x^{1/r}\cdot z,x^{1/r}\cdot z^{-1},x^{1/r}]][x^{-1/r}]\\
 & = & \left\{ \sum _{m,n}c_{m,n}x^{m/r}z^{n}:c_{m,n}\neq 0\Rightarrow |n|\leq m+O(1)\right\} 
\end{eqnarray*}
is an  \( L \)-algebra which contains  \( \widehat{M}((x^{1/r})) \).

Since  \( N(x,z) \) splits into linear factors in  \( L((x^{1/r}))[z] \),  \( N(x,z)=l\prod ^{s}_{i=1}(z-\zeta _{i}(x))^{k_{i}} \), there exists a partial
fraction decomposition of  \( h(x,z)=\frac{P(xz,\frac{x}{z})}{Q(xz,\frac{x}{z})}=\frac{z^{d}P(xz,\frac{x}{z})}{N(x,z)} \) in  \( \widehat{M}\ll x,z\gg  \):


\[
h(x,z)=\sum ^{l}_{j=0}P_{j}(x)z^{j}+\sum ^{s}_{i=1}\sum ^{k_{i}}_{k=1}\frac{P_{i,k}(x)}{(z-\zeta _{i}(x))^{k}}\]
with  \( P_{j}(x),P_{i,k}(x)\in \widehat{M}((x^{1/r})) \).

Recall that we are looking for the coefficient of  \( z^{0} \) in  \( h(x,z) \). We compute
it separately for each summand.

If  \( \zeta _{i}(x)=ax^{m/r}+... \) with  \( a\in L\setminus \{0\} \),  \( m>0 \), or  \( \zeta _{i}(x)=0 \), we have


\begin{eqnarray*}
\frac{1}{(z-\zeta _{i}(x))^{k}} & = & \frac{1}{z^{k}}\cdot \frac{1}{\left( 1-\frac{\zeta _{i}(x)}{z}\right) ^{k}}\\
 & = & \frac{1}{z^{k}}\cdot \sum ^{\infty }_{j=0}{k-1+j\choose k-1}\left( \frac{\zeta _{i}(x)}{z}\right) ^{j}\\
 & = & \sum ^{\infty }_{j=0}{k-1+j\choose k-1}\frac{\zeta _{i}(x)^{j}}{z^{k+j}}
\end{eqnarray*}
hence the coefficient of  \( z^{0} \) in  \( \frac{P_{i,k}(x)}{(z-\zeta _{i}(x))^{k}} \) is  \( 0 \).

If  \( \zeta _{i}(x)=ax^{m/r}+... \) with  \( a\in L\setminus \{0\} \),  \( m<0 \), we have
\[
\frac{1}{(z-\zeta _{i}(x))^{k}}=\frac{1}{(-\zeta _{i}(x))^{k}}\cdot \frac{1}{\left( 1-\frac{z}{\zeta _{i}(x)}\right) ^{k}}=\frac{1}{(-\zeta _{i}(x))^{k}}\cdot \sum _{j=0}^{\infty }{k-1+j\choose k-1}\left( \frac{z}{\zeta _{i}(x)}\right) ^{j}\]
hence the coefficient of  \( z^{0} \) in  \( \frac{P_{i,k}(x)}{(z-\zeta _{i}(x))^{k}} \) is  \( \frac{P_{i,k}(x)}{(-\zeta _{i}(x))^{k}} \).

The case  \( \zeta _{i}(x)=ax^{m/r}+... \) with  \( a\in L\setminus \{0\} \),  \( m=0 \), cannot occur, because it would imply  \( 0=N(0,\zeta _{i}(0))=N(0,a)=a^{d}. \)

Altogether we have
\[
g(x^{2})=[z^{0}]h(x,z)=P_{0}(x)+\sum _{\frac{1}{\zeta _{i}(x)}=o(x)}\sum ^{k_{i}}_{k=1}\frac{P_{i,k}(x)}{(-\zeta _{i}(x))^{k}}\in \widehat{M}((x^{1/r}))\]


Since all  \( \zeta _{i}(x) \)(in  \( L((x^{1/r})) \)) and all  \( P_{j}(x),P_{i,k}(x) \) (in  \( \widehat{M}((x^{1/r})) \)) are algebraic over  \( K(x) \), the same
holds also for  \( g(x^{2}) \). Hence  \( g(x) \) is algebraic over  \( K(x^{1/2}) \), hence also over  \( K(x) \).
After clearing denominators, we finally conclude that  \( g(x) \) is algebraic
over  \( R[x] \).

\begin{lyxsectionbibliography}

\item [1] Bruno Haible: D-finite power series in several variables. \em Diploma
thesis, University of Karlsruhe, June 1989. \em Sections 2.18 and
2.20.

\item [2] M. L. J. Hautus, D. A. Klarner: The diagonal of a double power
series. \em Duke Math. J. \em \bfseries 38 \mdseries (1971),
229-235.

\item [3] C. Chevalley: Introduction to the theory of algebraic functions
of one variable. \em Mathematical Surveys VI. American Mathematical
Society.\em 

\item [4] Jean-Pierre Serre: Corps locaux. \em Hermann. Paris \em 1968.

\item [5] Martin Eichler: Introduction to the theory of algebraic numbers
and functions. \em Academic Press. New York, London \em 1966.

\end{lyxsectionbibliography}

\end{document}
