%% This LaTeX-file was created by <bruno> Sun Feb 16 14:24:52 1997
%% LyX 0.10 (C) 1995 1996 by Matthias Ettrich and the LyX Team

%% Don't edit this file unless you are sure what you are doing.
\documentclass[12pt,a4paper,oneside,onecolumn]{article}
\usepackage[]{fontenc}
\usepackage[latin1]{inputenc}
\usepackage[dvips]{epsfig}

%%
%% BEGIN The lyx specific LaTeX commands.
%%

\makeatletter
\def\LyX{L\kern-.1667em\lower.25em\hbox{Y}\kern-.125emX\spacefactor1000}%
\newcommand{\lyxtitle}[1] {\thispagestyle{empty}
\global\@topnum\z@
\section*{\LARGE \centering \sffamily \bfseries \protect#1 }
}
\newcommand{\lyxline}[1]{
{#1 \vspace{1ex} \hrule width \columnwidth \vspace{1ex}}
}
\newenvironment{lyxsectionbibliography}
{
\section*{\refname}
\@mkboth{\uppercase{\refname}}{\uppercase{\refname}}
\begin{list}{}{
\itemindent-\leftmargin
\labelsep 0pt
\renewcommand{\makelabel}{}
}
}
{\end{list}}
\newenvironment{lyxchapterbibliography}
{
\chapter*{\bibname}
\@mkboth{\uppercase{\bibname}}{\uppercase{\bibname}}
\begin{list}{}{
\itemindent-\leftmargin
\labelsep 0pt
\renewcommand{\makelabel}{}
}
}
{\end{list}}
\def\lxq{"}
\newenvironment{lyxcode}
{\list{}{
\rightmargin\leftmargin
\raggedright
\itemsep 0pt
\parsep 0pt
\ttfamily
}%
\item[]
}
{\endlist}
\newcommand{\lyxlabel}[1]{#1 \hfill}
\newenvironment{lyxlist}[1]
{\begin{list}{}
{\settowidth{\labelwidth}{#1}
\setlength{\leftmargin}{\labelwidth}
\addtolength{\leftmargin}{\labelsep}
\renewcommand{\makelabel}{\lyxlabel}}}
{\end{list}}
\newcommand{\lyxletterstyle}{
\setlength\parskip{0.7em}
\setlength\parindent{0pt}
}
\newcommand{\lyxaddress}[1]{
\par {\raggedright #1 
\vspace{1.4em}
\noindent\par}
}
\newcommand{\lyxrightaddress}[1]{
\par {\raggedleft \begin{tabular}{l}\ignorespaces
#1
\end{tabular}
\vspace{1.4em}
\par}
}
\newcommand{\lyxformula}[1]{
\begin{eqnarray*}
#1
\end{eqnarray*}
}
\newcommand{\lyxnumberedformula}[1]{
\begin{eqnarray}
#1
\end{eqnarray}
}
\makeatother

%%
%% END The lyx specific LaTeX commands.
%%

\pagestyle{plain}
\setcounter{secnumdepth}{3}
\setcounter{tocdepth}{3}

%% Begin LyX user specified preamble:
\catcode`@=11 % @ ist ab jetzt ein gewoehnlicher Buchstabe
\def\mod#1{\allowbreak \mkern8mu \mathop{\operator@font mod}\,\,{#1}}
\def\pmod#1{\allowbreak \mkern8mu \left({\mathop{\operator@font mod}\,\,{#1}}\right)}
\catcode`@=12 % @ ist ab jetzt wieder ein Sonderzeichen


%% End LyX user specified preamble.
\begin{document}

The Legendre polynomials  \( P_{n}(x) \) are defined through 
\[
P_{n}(x)=\frac{1}{2^{n}n!}\cdot \left( \frac{d}{dx}\right) ^{n}(x^{2}-1)^{n}\]
(For a motivation
of the  \( 2^{n} \) in the denominator, look at  \( P_{n}(x) \) modulo an odd prime  \( p \), and
observe that  \( P_{n}(x)\equiv P_{p-1-n}(x)\mod p \) for  \( 0\leq n\leq p-1 \). This wouldn't hold if the  \( 2^{n} \) factor in the denominator
weren't present.)

\begin{description}

\item [Theorem:]~

\end{description}

 \( P_{n}(x) \) satisfies the recurrence relation


\[
P_{0}(x)=1\]



\[
(n+1)\cdot P_{n+1}(x)=(2n+1)x\cdot P_{n}(x)-n\cdot P_{n-1}(x)\]
for  \( n\geq 0 \) and the differential equation  \( (1-x^{2})\cdot P_{n}^{''}(x)-2x\cdot P_{n}^{'}(x)+(n^{2}+n)\cdot P_{n}(x)=0 \) for all  \( n\geq 0 \).

\begin{description}

\item [Proof:]~

\end{description}

Let  \( F:=\sum ^{\infty }_{n=0}P_{n}(x)\cdot z^{n} \) be the generating function of the sequence of polynomials. It
is the diagonal series of the power series
\[
G:=\sum _{m,n=0}^{\infty }\frac{1}{2^{n}m!}\cdot \left( \frac{d}{dx}\right) ^{m}(x^{2}-1)^{n}\cdot y^{m}\cdot z^{n}\]
Because the Taylor series
development theorem holds in formal power series rings (see [1], section
2.16), we can simplify
\begin{eqnarray*}
G & = & \sum _{n=0}^{\infty }\frac{1}{2^{n}}\cdot \left( \sum _{m=0}^{\infty }\frac{1}{m!}\cdot \left( \frac{d}{dx}\right) ^{m}(x^{2}-1)^{n}\cdot y^{m}\right) \cdot z^{n}\\
 & = & \sum _{n=0}^{\infty }\frac{1}{2^{n}}\cdot \left( (x+y)^{2}-1\right) ^{n}\cdot z^{n}\\
 & = & \frac{1}{1-\frac{1}{2}\left( (x+y)^{2}-1\right) z}
\end{eqnarray*}
We take over the terminology from the ``diag\_rational''
paper; here  \( R=Q[x] \) and  \( M=Q[[x]] \) (or, if you like it better,  \( M=H(C) \), the algebra of
functions holomorphic in the entire complex plane).  \( G\in M[[y,z]] \) is rational;
hence  \( F \) is algebraic over  \( R[z] \). Let's proceed exactly as in the ``diag\_series''
paper.  \( F(z^{2}) \) is the coefficient of  \( t^{0} \) in
\[
G(zt,\frac{z}{t})=\frac{2t}{2t-\left( (x+zt)^{2}-1\right) z}=\frac{2t}{-z^{3}\cdot t^{2}+2(1-xz^{2})\cdot t+(z-x^{2}z)}\]
The splitting field of the denominator
is  \( L=Q(x)(z)(\alpha ) \) where 
\[
\alpha _{1/2}=\frac{1-xz^{2}\pm \sqrt{1-2xz^{2}+z^{4}}}{z^{3}}\]

\[
\alpha =\alpha _{1}=\frac{2}{z^{3}}-\frac{2x}{z}+\frac{1-x^{2}}{2}z+\cdots \in Q(x)[[z]][\frac{1}{z}]\]

\[
\alpha _{2}=\frac{x^{2}-1}{2}z+\cdots \in Q(x)[[z]][\frac{1}{z}]\]
The partial fraction decomposition of  \( G(zt,\frac{z}{t}) \) reads
\[
G(zt,\frac{z}{t})=-\frac{2}{z^{3}}\cdot \frac{1}{\alpha _{1}-\alpha _{2}}\cdot \left( \frac{\alpha _{1}}{t-\alpha _{1}}-\frac{\alpha _{2}}{t-\alpha _{2}}\right) \]
It follows
that
\[
F(z^{2})=-\frac{2}{z^{3}}\cdot \frac{1}{\alpha _{1}-\alpha _{2}}\cdot \left( \frac{\alpha _{1}}{0-\alpha _{1}}-0\right) =\frac{1}{\sqrt{1-2xz^{2}+z^{4}}}\]
hence
\[
F(z)=\frac{1}{\sqrt{1-2xz+z^{2}}}\]


It follows that  \( (1-2xz+z^{2})\cdot \frac{d}{dz}F+(z-x)\cdot F=0 \). This is equivalent to the claimed recurrence.

Starting from the closed form for  \( F \), we compute a linear relation
for the partial derivatives of  \( F \). Write  \( \partial _{x}=\frac{d}{dx} \) and  \( \Delta _{z}=z\frac{d}{dz} \). One computes
\[
F=1\cdot F\]

\[
\left( 1-2xz+z^{2}\right) \cdot \partial _{x}F=z\cdot F\]

\[
\left( 1-2xz+z^{2}\right) ^{2}\cdot \partial _{x}^{2}F=3z^{2}\cdot F\]

\[
\left( 1-2xz+z^{2}\right) \cdot \Delta _{z}F=(xz-z^{2})\cdot F\]

\[
\left( 1-2xz+z^{2}\right) ^{2}\cdot \partial _{x}\Delta _{z}F=(z+xz^{2}-2z^{3})\cdot F\]

\[
\left( 1-2xz+z^{2}\right) ^{2}\cdot \Delta _{z}^{2}F=\left( xz+(x^{2}-2)z^{2}-xz^{3}+z^{4}\right) \cdot F\]
Solve
a homogeneous  \( 5\times 6 \) system of linear equations over  \( Q(x) \) to get 
\[
\left( 1-2xz+z^{2}\right) ^{2}\cdot \left( (-2x)\cdot \partial _{x}F+(1-x^{2})\cdot \partial _{x}^{2}F+\Delta _{z}F+\Delta _{z}^{2}F\right) =0\]
Divide by
the first factor to get
\[
(-2x)\cdot \partial _{x}F+(1-x^{2})\cdot \partial _{x}^{2}F+\Delta _{z}F+\Delta _{z}^{2}F=0\]
This is equivalent to the claimed equation
 \( (1-x^{2})\cdot P_{n}^{''}(x)-2x\cdot P_{n}^{'}(x)+(n^{2}+n)\cdot P_{n}(x)=0 \).

\begin{lyxsectionbibliography}

\item [1] Bruno Haible: D-finite power series in several variables. \em Diploma
thesis, University of Karlsruhe, June 1989\em . Sections 2.14, 2.15
and 2.22.

\end{lyxsectionbibliography}

\end{document}
