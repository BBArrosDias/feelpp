%% This LaTeX-file was created by <bruno> Sun Feb 16 14:06:08 1997
%% LyX 0.10 (C) 1995 1996 by Matthias Ettrich and the LyX Team

%% Don't edit this file unless you are sure what you are doing.
\documentclass[12pt,a4paper,oneside,onecolumn]{article}
\usepackage[]{fontenc}
\usepackage[latin1]{inputenc}
\usepackage[dvips]{epsfig}

%%
%% BEGIN The lyx specific LaTeX commands.
%%

\makeatletter
\def\LyX{L\kern-.1667em\lower.25em\hbox{Y}\kern-.125emX\spacefactor1000}%
\newcommand{\lyxtitle}[1] {\thispagestyle{empty}
\global\@topnum\z@
\section*{\LARGE \centering \sffamily \bfseries \protect#1 }
}
\newcommand{\lyxline}[1]{
{#1 \vspace{1ex} \hrule width \columnwidth \vspace{1ex}}
}
\newenvironment{lyxsectionbibliography}
{
\section*{\refname}
\@mkboth{\uppercase{\refname}}{\uppercase{\refname}}
\begin{list}{}{
\itemindent-\leftmargin
\labelsep 0pt
\renewcommand{\makelabel}{}
}
}
{\end{list}}
\newenvironment{lyxchapterbibliography}
{
\chapter*{\bibname}
\@mkboth{\uppercase{\bibname}}{\uppercase{\bibname}}
\begin{list}{}{
\itemindent-\leftmargin
\labelsep 0pt
\renewcommand{\makelabel}{}
}
}
{\end{list}}
\def\lxq{"}
\newenvironment{lyxcode}
{\list{}{
\rightmargin\leftmargin
\raggedright
\itemsep 0pt
\parsep 0pt
\ttfamily
}%
\item[]
}
{\endlist}
\newcommand{\lyxlabel}[1]{#1 \hfill}
\newenvironment{lyxlist}[1]
{\begin{list}{}
{\settowidth{\labelwidth}{#1}
\setlength{\leftmargin}{\labelwidth}
\addtolength{\leftmargin}{\labelsep}
\renewcommand{\makelabel}{\lyxlabel}}}
{\end{list}}
\newcommand{\lyxletterstyle}{
\setlength\parskip{0.7em}
\setlength\parindent{0pt}
}
\newcommand{\lyxaddress}[1]{
\par {\raggedright #1 
\vspace{1.4em}
\noindent\par}
}
\newcommand{\lyxrightaddress}[1]{
\par {\raggedleft \begin{tabular}{l}\ignorespaces
#1
\end{tabular}
\vspace{1.4em}
\par}
}
\newcommand{\lyxformula}[1]{
\begin{eqnarray*}
#1
\end{eqnarray*}
}
\newcommand{\lyxnumberedformula}[1]{
\begin{eqnarray}
#1
\end{eqnarray}
}
\makeatother

%%
%% END The lyx specific LaTeX commands.
%%

\pagestyle{plain}
\setcounter{secnumdepth}{3}
\setcounter{tocdepth}{3}

%% Begin LyX user specified preamble:
\catcode`@=11 % @ ist ab jetzt ein gewoehnlicher Buchstabe
\def\ll{\langle\!\langle}
\def\gg{\rangle\!\rangle}
\catcode`@=12 % @ ist ab jetzt wieder ein Sonderzeichen


%% End LyX user specified preamble.
\begin{document}

The Laguerre polynomials  \( L_{n}(x) \) are defined through 
\[
L_{n}(x)=e^{x}\cdot \left( \frac{d}{dx}\right) ^{n}(x^{n}e^{-x})\]


\begin{description}

\item [Theorem:]~

\end{description}

 \( L_{n}(x) \) satisfies the recurrence relation


\[
L_{0}(x)=1\]



\[
L_{n+1}(x)=(2n+1-x)\cdot L_{n}(x)-n^{2}\cdot L_{n-1}(x)\]
for  \( n\geq 0 \) and the differential equation  \( x\cdot L_{n}^{''}(x)+(1-x)\cdot L_{n}^{'}(x)+n\cdot L_{n}(x)=0 \) for all  \( n\geq 0 \).

\begin{description}

\item [Proof:]~

\end{description}

Let  \( F:=\sum ^{\infty }_{n=0}\frac{L_{n}(x)}{n!}\cdot z^{n} \) be the exponential generating function of the sequence of polynomials.
It is the diagonal series of the power series
\[
G:=\sum _{m,n=0}^{\infty }\frac{1}{m!}\cdot e^{x}\cdot \left( \frac{d}{dx}\right) ^{m}(x^{n}e^{-x})\cdot y^{m}\cdot z^{n}\]
Because the Taylor series
development theorem holds in formal power series rings (see [1], section
2.16), we can simplify
\begin{eqnarray*}
G & = & e^{x}\cdot \sum _{n=0}^{\infty }\left( \sum _{m=0}^{\infty }\frac{1}{m!}\cdot \left( \frac{d}{dx}\right) ^{m}(x^{n}e^{-x})\cdot y^{m}\right) \cdot z^{n}\\
 & = & e^{x}\cdot \sum _{n=0}^{\infty }(x+y)^{n}e^{-(x+y)}\cdot z^{n}\\
 & = & \frac{e^{-y}}{1-(x+y)z}
\end{eqnarray*}
We take over the terminology from the ``diag\_rational''
paper; here  \( R=Q[x] \) and  \( M=Q[[x]] \) (or, if you like it better,  \( M=H(C) \), the algebra of
functions holomorphic in the entire complex plane).  \( G\in M[[y,z]] \) is not rational;
nevertheless we can proceed similarly to the ``diag\_series'' paper.
 \( F(z^{2}) \) is the coefficient of  \( t^{0} \) in
\[
G(zt,\frac{z}{t})=\frac{e^{-zt}}{1-z^{2}-\frac{xz}{t}}\in M[[zt,\frac{z}{t},z]]=M\ll z,t\gg \]
The denominator's only zero is  \( t=\frac{xz}{1-z^{2}} \). We
can write
\[
e^{-zt}=e^{-\frac{xz^{2}}{1-z^{2}}}+\left( zt-\frac{xz^{2}}{1-z^{2}}\right) \cdot P(z,t)\]
with  \( P(z,t)\in Q[[zt,\frac{xz^{2}}{1-z^{2}}]]\subset Q[[zt,x,z]]=M[[zt,z]]\subset M\ll z,t\gg  \). This yields -- all computations being done in  \( M\ll z,t\gg  \)
--
\begin{eqnarray*}
G(zt,\frac{z}{t}) & = & \frac{e^{-\frac{xz^{2}}{1-z^{2}}}}{1-z^{2}-\frac{xz}{t}}+\frac{zt}{1-z^{2}}\cdot P(z,t)\\
 & = & \frac{1}{1-z^{2}}\cdot e^{-\frac{xz^{2}}{1-z^{2}}}\cdot \sum _{j=0}^{\infty }\left( \frac{x}{1-z^{2}}\frac{z}{t}\right) ^{j}+\frac{zt}{1-z^{2}}\cdot P(z,t)
\end{eqnarray*}
Here, the coefficient of  \( t^{0} \) is
\[
F(z^{2})=\frac{1}{1-z^{2}}\cdot e^{-\frac{xz^{2}}{1-z^{2}}}\]
hence
\[
F(z)=\frac{1}{1-z}\cdot e^{-\frac{xz}{1-z}}\]


It follows that  \( (1-z)^{2}\cdot \frac{d}{dz}F-(1-x-z)\cdot F=0 \). This is equivalent to the claimed recurrence.

Starting from the closed form for  \( F \), we compute a linear relation
for the partial derivatives of  \( F \). Write  \( \partial _{x}=\frac{d}{dx} \) and  \( \Delta _{z}=z\frac{d}{dz} \). One computes
\[
F=1\cdot F\]

\[
\left( 1-z\right) \cdot \partial _{x}F=-z\cdot F\]

\[
\left( 1-z\right) ^{2}\cdot \partial _{x}^{2}F=z^{2}\cdot F\]

\[
\left( 1-z\right) ^{2}\cdot \Delta _{z}F=((1-x)z-z^{2})\cdot F\]

\[
\left( 1-z\right) ^{3}\cdot \partial _{x}\Delta _{z}F=(-z+xz^{2}+z^{3})\cdot F\]
Solve
a homogeneous  \( 4\times 5 \) system of linear equations over  \( Q(x) \) to get 
\[
\left( 1-z\right) ^{3}\cdot \left( (1-x)\cdot \partial _{x}F+x\cdot \partial _{x}^{2}F+\Delta _{z}F\right) =0\]
Divide by
the first factor to get
\[
(1-x)\cdot \partial _{x}F+x\cdot \partial _{x}^{2}F+\Delta _{z}F=0\]
This is equivalent to the claimed equation
 \( x\cdot L_{n}^{''}(x)+(1-x)\cdot L_{n}^{'}(x)+n\cdot L_{n}(x)=0 \).

\begin{lyxsectionbibliography}

\item [1] Bruno Haible: D-finite power series in several variables. \em Diploma
thesis, University of Karlsruhe, June 1989\em . Sections 2.15 and
2.22.

\end{lyxsectionbibliography}

\end{document}
