% (C) 2013 - Université de Strasbourg
% * Guillaume Dollé <guillaume.dolle@math.unistra.fr>
% * Christophe Prud'homme <christophe.prudhomme@feelpp.org>
% Tutorial documentation - myapp
%


\section{First \feel Application}
\label{sec:myapp}

See section \ref{chap:tutorial-building} for more information about \feel installation.

\subsection{Minimal example}

Let's begin with our first program using the \feel framework
(source \textcolor{magenta}{"doc/manual/tutorial/myapp.cpp"}).
Before all, you have to include the \feel headers.
%
\vspace{2mm}
\begin{lstlisting}
  #include <feel/feel.hpp>
  using namespace Feel;
\end{lstlisting}
\vspace{2mm}
%
We use the C++ \lstinline!namespace! to avoid \lstinline!Feel::! prefix before
\feel objects.
%
\vspace{2mm}
\lstinputlisting[linerange=marker1-endmarker1]{tutorial/myapp.cpp}
\vspace{2mm}
%
\begin{itemize}
\item
We pass command line options using the
\href{http://www.boost.org/doc/libs/1_53_0/doc/html/program_options.html}
{Boost Program Options}\footnotemark[1] library using the prefix \lstinline!po::! which is 
a \feel alias for the Boost::program\_options namespace.
%
\footnotetext[1]{\url{http://www.boost.org/doc/libs/1_53_0/doc/html/program_options.html}}
%
To add a new \feel option, we must create a new
\feel \lstinline!options_description!. You must add the default \feel options
and the new one that we choose here as a double value. Note that the default
value will be assigned if not specified by the user.

\item
Then we initialize the environment variables through the \feel
\lstinline!Environment! class (Check the Constructor prototype on the online documentation).

\item
We instantiate a new application. We specify the directory where to execute the
program. That could be usefull for archiving your results.

\item
Finally, we save the results in a log file using the
\href{http://code.google.com/p/google-glog/}{google-glog library}
\footnotemark[2].
%
\footnotetext{\url{http://code.google.com/p/google-glog/}}
%
As you can see, we save in this example our custom option value and the
current processor number.

\end{itemize}


\subsection{Compilation, execution, logs}

To compile a tutorial, just use the GNU make command.
%
\begin{unixcom}
    make feelpp_doc_<appname>
\end{unixcom}
%
where \textit{appname} is the name of the application you wish to
compile (here, \lstinline!myapp!).
Go to the execution directory as specified in the program,
and execute it. You can change your option value.
%
\begin{unixcom}
    ./feelpp_doc_myapp [--value 6.6]
\end{unixcom}
%
You can list the log files created.
%
\begin{unixcom}
    ls /tmp/<your login>/feelpp_doc_myapp/
\end{unixcom}
%
If you open one of these log,
you should be able to see your value and the processor number 
used to compute.
You can run your application on several processors using MPI.
%
\begin{unixcom}
    mpirun -np 2 feelpp_doc_myapp
\end{unixcom}
%
Note that there will be one log for each processor in that case.


\subsection{Config files}

A config file can be parsed to the program to profile your options.
The default config paths are,
\begin{enumerate}
    \item current dir
    \item \verb|$HOME/feel/config/|
    \item \verb|$INSTALL_PREFIX/share/feel/config/|
\end{enumerate}
then you have to write inside one of these folders a file called
\lstinline!<app_name>.cfg! or \lstinline!feelpp_<app_name>.cfg!.
For example, our \lstinline!myapp.cfg! would looks like,
%
\vspace{2mm}
\begin{lstlisting}
    value=0.53
\end{lstlisting}
\vspace{2mm}
%
Note that you can specify the config file through the option \lstinline!--config-file=<path>!

