% (C) 2013 - Université de Strasbourg
% * Guillaume Dollé <guillaume.dolle@math.unistra.fr>
% * Christophe Prud'homme <christophe.prudhomme@feelpp.org>
% Tutorial documentation - myintegrals
%

\section{Computing Integrals}
\label{sec:myintegrals}

You should be able to create a mesh now. If it is not the case, get back to the
section \ref{sec:mymesh}. This part explains how to integrate on a mesh with \feel
(source \textcolor{magenta}{"doc/manual/tutorial/myintegrals.cpp"}).
Let's consider the domain $\Omega=[0,1]^d$ and associated meshes.
Here, we want to integrate the following function,
%
\begin{equation}
    f(x,y,z) = x^2 + y^2 + z^2
\end{equation}
%
on the whole domain $\Omega$ and on part of the boundary $\Omega$. Take a look at the code.
%
\vspace{2mm}
\lstinputlisting[linerange=marker_main-endmarker_main]{tutorial/myintegrals.cpp}
\vspace{2mm}
%
To use the \lstinline!integrate()! function, we have to precise the domain range. You can use,
\begin{itemize}
    \item \lstinline!elements()! to iterate on the whole mesh $\Omega$,
    \item \lstinline!boundaryfaces()! to iterate on the boundary $\partial\Omega$,
    \item \lstinline!markedfaces()! to iterate on a choose face.
\end{itemize}
%
You have to specify the expression we wish to compute. \feel provides a set of functions
to write these expressions \ref{sec:keywords}.
The \lstinline!evaluate()! function computes the integral on the global mesh.
The \lstinline!false! parameter limits the computation on the subregion owned by
the processor.
%
Note that \feel computes automatically the quadrature and consider by default each
non polynomial terms of the expression as a polynomial of degree 2. You can change
it by passing a \lstinline!_quad! parameter to the \lstinline!integrate()! function 
which takes a \lstinline!_Q<int order>! object as value.
(refer to API documentation).



