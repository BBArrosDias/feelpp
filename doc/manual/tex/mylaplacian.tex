% (C) 2013 - Université de Strasbourg
% * Guillaume Dollé <guillaume.dolle@math.unistra.fr>
% * Christophe Prud'homme <christophe.prudhomme@feelpp.org>
% Tutorial documentation - mymesh
%


\section{Laplacian Problem}
\label{sec:tuto-mylaplacian}

This part explains how to solve the Laplacian equation 
for homogeneous dirichlet conditions,
%
\begin{equation}
\left\{
\begin{array}{rcll}
    -\Delta u & = & f & \text{on}\;\Omega \;, \\
            u & = & 0 & \text{on}\;\partial\Omega \;,\\
\end{array}
\right.
\label{eq:tuto-mylaplacian-prob}
\end{equation}
%
where $u\in\Omega$ is the unknown "trial" function and $\Omega$ the domain.
%
We multiply each part of the first equation by a "test" function $v\in H_0^1(\Omega)$ 
and we integrate the resulting equation on the domain $\Omega$,
%
\begin{equation}
-\int_\Omega \Delta u v = \int_\Omega f v \;.
\end{equation}
%
We can integrate by parts this equation (Green Theorem) to obtain \underline{the variationnal
formulation},
%
\begin{equation}
\int_\Omega \nabla u \nabla v
-\underbrace{ \int_{\partial\Omega} \frac{\partial u}{\partial n} v }_{= 0}
=\int_\Omega f v \;,
\end{equation}
%
where $n$ denotes a unit outward normal vector to the boundary. We can rewrite the problem
(\ref{eq:tuto-mylaplacian-prob}) as find $u\in H_0^1(\Omega)$ such that for all
$v\in H_0^1(\Omega)$,
%
\begin{equation}
a(u,v)=l(v) \;,
\end{equation}
where $a$ is a bilinear form, continuous, coercive and $l$ a linear form.
Let's take a look at the \feel code
(source \textcolor{magenta}{"doc/manual/tutorial/mylaplacian.cpp"}).
We consider for this example $f=1$ constant.
%
\vspace{2mm}
\lstinputlisting[linerange=marker_main-endmarker_main]{tutorial/mylaplacian.cpp}
\vspace{2mm}
%
As you can see, the program looks very close to the mathematical formulation.
We use the \lstinline!form2()! function to define the bilinear form and \lstinline!form1()!
for the linear one. The gradient for the trial functions is declared with the \lstinline!gradt()!
expression where as \lstinline!grad()! is used for the test functions
(see all keywords \ref{sec:keywords}).
Note that we need to transpose the second vector to perform the scalar product.
To introduce the homogeneous dirichlet conditions on the boundary, we use the function
\lstinline!on()!. Once the variationnal formulation and the boundary conditions are set, we call
the solver with \lstinline!solve()!.






