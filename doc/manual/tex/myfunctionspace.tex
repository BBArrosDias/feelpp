% (C) 2013 - Université de Strasbourg
% * Guillaume Dollé <guillaume.dolle@math.unistra.fr>
% * Christophe Prud'homme <christophe.prudhomme@feelpp.org>
% Tutorial documentation - myfunctionspace
%


\section{Function Spaces}
\label{sec:myfunctionspace}

Now we are able to construct basic \feel applications and compute some integrals 
(If it is not the case, get back to section \ref{sec:mymesh}).
We interest now to solve partial differential equations, so we must define function spaces
to work on. (source \textcolor{magenta}{"doc/manual/tutorial/myfunctionspace.cpp"}).
%
\vspace{2mm}
\lstinputlisting[linerange=marker_main-endmarker_main]{tutorial/myfunctionspace.cpp}
\vspace{2mm}
%
As you can see, we instantiate a new function space object
$X_h\subset Pc_h^k=\{v\in C^0(\Omega) \;|\; \forall K\in \tau_h,  v_{|_K}\in P_k(K)\}$ using the \lstinline!Pch<int K>()!
template. These functions are continuous piecewise polynomials of an order K and the basis functions are
Lagrange polynomials.
Let $g$ be a function such as for all $(x,y,z)\in \Omega$ we have,
%
\[
    g(x,y,z)= \sin(\frac{\pi x}{2}).\cos(\frac{\pi y}{2}).\cos(\frac{\pi z}{2}) \;
\]
%
We build the Lagrange interpolant $u\in X_h$ of $g$ and $w\in X_h$ of $u-g$. We use for this the
\lstinline!vf::project()! function  which returns the projection on the mesh node for
an expression given as parameter. Then we compute the L2 norm,
%
\[
    \| g \|_2 =
    \sqrt{\left( \int_\Omega g^2 \right)} \;,
\]
%
and the error $\| g - g_{X_h}\|_2$
using the \lstinline!normL2()! function.
Finally, we export our elements $u,w\in X_h$ to visualize them with paraview.
To do so, we have to add them to the exporter.

