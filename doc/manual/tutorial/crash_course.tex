\feelchapter{Crash course}
            {Crash course}
            {Vincent Huber}
            {cha:tutorial-crash-course}

This chapter is designed for impatient people who wants to test \feel as soon as possible.
\section{Requirements}
Before installing \feel, you need to get this \underline{required packages:}\\
\begin{itemize}
\item g++ ($4.7$ or $4.8$) OR Clang ($\geq 3.3$)
\item MPI : openmpi (preferred) or mpich
\item Boost ($\geq$1.39)
\item Petsc ($\geq$2.3.3)
\item Cmake ($\geq$2.6)
\item Gmsh\footnote{Gmsh is a pre/post processing software for scientific
computing available at \url{http://www.geuz.org/gmsh}}
\item Libxml2
\end{itemize}
It is assumed that all this packages are properly installed.

\section{Building \feel from source on *nix}
\feel is distributed as a tarball once in a while. The tarballs are available
at
\begin{center}
  \href{http://code.google.com/p/feelpp/downloads/list}{http://code.google.com/p/feelpp/downloads/list}
\end{center}
Download the latest tarball. Then follow the steps and replace
\texttt{x},\texttt{y},\texttt{z} with the corresponding numbers

\begin{unixcom}
  tar xzf feel-x.y.z.tar.gz
  cd feel-x.y.z
\end{unixcom}
We define then the current directory as the source one, ie:
\begin{unixcom}
  export FeelppSrcDir=`pwd`
\end{unixcom}

\section{Compiling}
Please, notice that 4 Gbytes of RAM is a minimum to make \feel compiling (with \textsc{gcc}), with clang ($\geq 3.1$), the memory footprint is much lower.

In order to compile \feel and a test application, we create a new directory:
\begin{unixcom}
  mkdir build
  cd build
  export buildDir=`pwd`
\end{unixcom}

and then, we are able to compile our first application:
\begin{unixcom}
  cd $buildDir
  cmake $FeelppSrcDir
  make -j4
\end{unixcom}

This procedure will build the entire librarie (with DebWithRelInfo option) and a dummy program to presents the \feel's abilities.

There is a procedure to install as a system librarie \feel.
\begin{unixcom}
  make install
\end{unixcom}
See chapter 2 for more details.

\section{\feel Hello World}
\label{sec:feel-hello-world}

As an introduction to the aim and the way to do with \feel, we provide a sort of
\textit{Hello World} program to evaluate the library.

\subsection{About the math}
\label{sec:about-math}

We want to solve the simplest problem:
\begin{equation}\nonumber
  \begin{aligned}
    - \Delta u &= 1,\\
    u_{|\partial \Omega} &= 0,
  \end{aligned}
\end{equation}
where $\Omega \in \mathbb{R}^n, n\in{1,2,3}$.\\

That problem written in variational form is:
looks for $v\in H^1\left( \Omega \right)$
\begin{equation}\nonumber
  \begin{aligned}
    a\left( u,v \right)&=l\left( v \right)\\
\forall v &\in H^1\left( \Omega \right).
  \end{aligned}
\end{equation}
with:
\begin{equation}\nonumber
  \begin{aligned}
    a\left( u,v \right)&=\int_{\Omega} \nabla u \cdot \nabla v ,\\
    l\left( v \right) &= \int_{\Omega} v .
  \end{aligned}
\end{equation}

The aim of \feel is to provide the simplest way to write the $a$ and $f$ forms.\\

From a discrete point of view, we introduce $V_h\subset H^1\left( \Omega \right)$ such that:
\begin{equation}\nonumber
  \begin{aligned}
V_h = \left\{ v \in C^0\left( \Omega \right), \forall K\in \mathcal{T}_h, \right.v\left|_K \in P_1\left( K \right) \right\},
    \end{aligned}
  \end{equation}
where $\mathcal{T}_h$ is the set of element $K$ forming the mesh of $\Omega$. \\
We now look for $u_h \in V_h$ such that:
\begin{equation}\nonumber
  \begin{aligned}
    \forall v_h\in V_h, a\left( u_h,v_h \right)=l\left( v_h \right).
    \end{aligned}
  \end{equation}

\subsection{About the code}
\label{sec:about-code}

This section is here to declare that we want to use the namespace \feel, to
passe the command line options to the created environnement and add some
informations (basics \feel options, application name).
\lstinputlisting[linerange=marker1-endmarker1]{../../quickstart/laplacian.cpp}
We have to define the mesh, the approximation space and our test and trial
functions.
\lstinputlisting[linerange=marker2-endmarker2]{../../quickstart/laplacian.cpp}
We create now our bilinear and linear forms, we add the homogeneous Dirichlet
conditions and solve the discretized (linear) system.
\lstinputlisting[linerange=marker3-endmarker3]{../../quickstart/laplacian.cpp}
\feel provides the possibility to save the results:
\lstinputlisting[linerange=marker4-endmarker4]{../../quickstart/laplacian.cpp}


\section{First execution \& vizualisation}
\label{sec:first-execution}

To test that part of code, please go to:
\begin{unixcom}
  cd $FeelppSrcDir/quickstart
\end{unixcom}
and execute the code, by:
\begin{unixcom}
  ./feelpp_qs_laplacian
\end{unixcom}
This will produce several files:
\begin{unixcom}
  qs_laplacian-1_0.case
  qs_laplacian-1.sos
  qs_laplacian.timeset
  qs_laplacian.u-1_0.001
  qs_laplacian-1_0.geo001
  qs_laplacian.pid-1_0.001
  square.geo
  square.msh
\end{unixcom}
You can vizualise the results using any Ensight file reader, such as Paraview,
opening \verb=qs_laplacian-1.sos=.
\begin{unixcom}
  paraview qs_laplacian-1.sos
\end{unixcom}
You may have a look to the differents option provided by
\begin{unixcom}
  ./feelpp_qs_laplacian --help
\end{unixcom}


%%% Local Variables:
%%% coding: utf-8
%%% mode: latex
%%% TeX-PDF-mode: t
%%% TeX-parse-self: t
%%% x-symbol-8bits: nil
%%% TeX-auto-regexp-list: TeX-auto-full-regexp-list
%%% TeX-master: "../feel-manual"
%%% ispell-local-dictionary: "american"
%%% End:

