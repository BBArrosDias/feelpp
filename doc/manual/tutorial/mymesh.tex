% (C) 2013 - Université de Strasbourg
% * Guillaume Dollé <guillaume.dolle@math.unistra.fr>
% * Christophe Prud'homme <christophe.prudhomme@feelpp.org>
% Tutorial documentation - mymesh
%


\section{Mesh Manipulation}
\label{sec:mymesh}

Feel++ provides some tools to manipulate mesh. 
Here is a basic example that shows
you how to generate a mesh for a square geometry
(source \textcolor{magenta}{"doc/manual/tutorial/mymesh.cpp"}).
%
\vspace{2mm}
\lstinputlisting[linerange=marker_main-endmarker_main]{tutorial/mymesh.cpp}
\vspace{2mm}

As always, we initialise the \feel environment (see section \ref{sec:myapp}).
The \lstinline!unitSquare()! will generate a mesh for a square geometry.
\feel provides several functions to automate the GMSH mesh generation
for different topologies.
%
( \lstinline!unitCircle()!,
  \lstinline!unitCube()!,
  \dots ).
%
These functions will create a geometry file
\textit{.geo} and a mesh file \textit{.msh}. We can visualize them in GMSH. 
%
\begin{unixcom}
    gmsh <entity_name>.msh
\end{unixcom}
%
Finally we use the \lstinline!exporter()! function to export the mesh for post processing.
It will create by default a \textbf{Paraview} format file \textit{.sos} and an \textbf{Ensight}
format file \textit{.case}.
%
\begin{unixcom}
    paraview <app_name>.sos
\end{unixcom}
%
For advanced usage, there is the more generic \lstinline!createGMSHMesh()! function which is
useful for creating or loading an existing mesh or geometry (see section \ref{howto:spec-meshes}
for a load example).
Note that \lstinline!unitSquare()! is just a particular case of \lstinline!createGMSHMesh()!.
\feel provide useful tools to iterate on the mesh or some faces that we will see later.
%
The process of the mesh creation is fully parallelized. You can as explained in section \ref{sec:myapp}
run this example on several processors and visualise subregions with paraview.


