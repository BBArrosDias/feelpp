% (C) 2013 - Université de Strasbourg
% * Guillaume Dollé <guillaume.dolle@math.unistra.fr>
% * Christophe Prud'homme <christophe.prudhomme@feelpp.org>
% Tutorial documentation - mymesh
%

\section{Advection problem}
\label{sec:tuto-myadvection}

The advection problem can be seen as a liquid polluant which is injected in a river.
The polluant is dragged by the river flow as it can be diffused depending on 
certain conditions. This can be described by an equation containing a diffusion, an advection and
a reaction term as follows,
%
\begin{equation}
\left\{
\begin{array}{rcll}
-\epsilon\Delta  u + \vec\beta \nabla  u + \mu u & = & f & \text{on}\; \Omega \;, \\
                         u  & = & 0 & \text{on}\; \partial\Omega \;, \\
\end{array}
\right.
\label{eq:tuto-adv}
\end{equation}
%
We use here homogeneous Dirichlet boundary conditions.

\subsubsection{Variationnal formulation}

To establish the variationnal formulation, as always we mutiply the first equation by a
test function $v\in H_0^1(\Omega)$ such that,
\[
    H_0^1(\Omega) = \{ v\in H^1(\Omega),\; v=0 \; \text{on} \; \partial\Omega \} \;.
\]
Then we integrate on the domain $\Omega$,
%
\begin{equation}
- \int_\Omega \epsilon\Delta u v
+ \int_\Omega \vec\beta \nabla u v
+ \int_\Omega \mu u v
= \int_\Omega f v \;.
\end{equation}
%
We establish the variationnal formulation from the previous equation and using the Green formula,
%
\begin{equation}
\int_\Omega \epsilon \nabla u \nabla v
- \underbrace{\int_{\partial\Omega} \epsilon (\nabla u \cdot \mathbf n) v}_{=0}
+ \int_\Omega \vec\beta \nabla u v
+ \int_\Omega \mu u v
= \int_\Omega f v \;,
\label{eq:tuto-adv-varform}
\end{equation}
%
where $\mathbf n$ is a unit outward normal vector. We can rewrite this equation such that,
%
\begin{equation}
    a(u,v) = l(v)
\label{eq:tuto-adv-bilform}
\end{equation}
%
where $a$ is a bilinear form, continuous, coercive and $l$ is a linear form.

\subsubsection{Application}

We choose for our example $\mu = 1$, $\epsilon = 1$, $f=1$, and
$\beta=(1,1)^T$.
See the corresponding code. 
(source \textcolor{magenta}{"doc/manual/tutorial/myadvection.cpp"}).
%
\vspace{2mm}
\lstinputlisting[linerange=marker_main-endmarker_main]{tutorial/myadvection.cpp}
\vspace{2mm}
%
The beauty of \feel is that while the variationnal formulation is correctly written
then the code is in most cases very close to the Mathematical statement.
Here again, we create the mesh for an unit square geometry. Then we define the function
space $X_h$ we choose as order 1 Lagrange basis function using \lstinline!Pch<Order>()!.
Note that here, the function space is the same for "trial" and "test" functions.
We declare the left and the right hand side integrals expressions for the equation
(\ref{eq:tuto-adv-varform}). Finally we add the Dirichlet boundary condition and we use
the default solver to solve (\ref{eq:tuto-adv-bilform}).
We export the solution $u$ for post processing.



