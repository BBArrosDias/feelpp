% (C) 2013 - Université de Strasbourg
% * Guillaume Dollé <guillaume.dolle@math.unistra.fr>
% * Christophe Prud'homme <christophe.prudhomme@feelpp.org>
% Tutorial documentation - myfunctionspace
%
\section{Function Spaces}
\label{sec:myfunctionspace}

Now we are able to construct basic \feel applications and compute some integrals 
(If it is not the case, get back to section \ref{sec:mymesh}).
We interest now to solve partial differential equations, so we must define function spaces
to work on. (source \textcolor{magenta}{"doc/manual/tutorial/myfunctionspace.cpp"}).

\vspace{2mm}
\lstinputlisting[linerange=marker_main-endmarker_main]{tutorial/myfunctionspace.cpp}
\vspace{2mm}

As you can see, we instanciate a new function space object $X_h$ using the \lstinline!Pch<int order>(mesh)!
template. These functions are piecewise polynomial of an order precised and the basis functions are
Lagrange polynomials. Then for $u,v\in X_h$, we build the Lagrange interpolant. We use for this
\lstinline!project()! which returns the projection on the mesh node.

test.


