% (C) 2013 - Université de Strasbourg
% * Vincent Huber <vincent.huber@cemosis.fr>
% * Guillaume Dollé <guillaume.dolle@math.unistra.fr>
% * Christophe Prud'homme <christophe.prudhomme@feelpp.org>
% Tutorial documentation - mymesh
%

\section{Stokes Problem}
\label{sec:tuto-mystokes}

Let solve the stokes equation considering a Poiseuille flow profile. 
We have the following system of equations,
%
\begin{equation}
\left\{
\begin{array}{rcll}
-\mu\Delta \bf u + \nabla p & = & \bf f & \text{on}\; \Omega \;, \\
                \div(\bf u) & = & 0 & \text{on}\; \Omega \;, \\
                \bf u  & = & g & \text{on}\; \Gamma \;, \\
\end{array}
\right.
\label{eq:tuto-stk}
\end{equation}
%
where $u\in [H_g^1(\Omega)]^d$ denotes the flow speed, $p\in [L_0^2(\Omega)]$ the fluid pressure, $\mu$ the 
fluid viscosity.
The last boundary condition expresses a null pressure fixed
on the outlet. The Poiseuille profile on the boundary is,
%
\begin{equation}
g(x,y)=
\left(
\begin{array}{c}
 y(1-y) \\
 0      \\
\end{array}
\right)
\end{equation}
%
The method used to obtain the strong formulation is closed to the one used
for the laplacian (see section \ref{sec:tuto-mylaplacian}).
We multiply the first equation by a test function $v\in H^1(\Omega)$
and we integrate on the domain $\Omega$,
%
\begin{equation}
-\int_\Omega \mu \Delta \mathbf u \cdot \mathbf v
+\int_\Omega \nabla p \cdot \mathbf v
=\int_\Omega \mathbf f \cdot \mathbf v \;.
\end{equation}
%
Then we use the Green Formula on the first term and we rewrite the second one 
to get the following equation,
%
\begin{equation}
\left(
\int_\Omega \mu \nabla \mathbf u : \nabla \mathbf v
-\int_{\partial\Omega} \frac{\partial \mathbf u}{\partial \mathbf n} \cdot \mathbf v
\right)
+\int_\Omega ( \div(p \mathbf v) - p\div(\mathbf v ) )
=\int_\Omega \mathbf f \cdot \mathbf v \;.
\label{eq:tuto-stk-1}
\end{equation}
%
where $n$ denotes a normal vector on the boundary.
The divergence theorem (or Gauss's theorem) gives,
%
\begin{equation}
\int_\Omega \div(p \mathbf v) = \int_{\partial\Omega} p \mathbf v\cdot \mathbf n \;.
\label{eq:tuto-stk-gauss}
\end{equation}
%
We have to add a consistency terms to the equation (\ref{eq:tuto-stk-1}) to
guaranty the symmetry of the bilinear form.
This term is provided by the second equation (\ref{eq:tuto-stk}). We multiply this equation
by a test function $q\in L_2(\Omega)$ and we integrate on the domain $\Omega$,
%
\begin{equation}
\int_{\Omega} \div(\mathbf u) q = 0 \;,
\label{eq:tuto-stk-secondeq}
\end{equation}
%
Finally, we deduce from the equations (\ref{eq:tuto-stk-gauss}), (\ref{eq:tuto-stk-secondeq})
and after rearranging the integrals (\ref{eq:tuto-stk-1}) the variationnal formulation,
%
\begin{equation}
\int_\Omega \mu \nabla \mathbf u :\nabla \mathbf v
+\int_\Omega \left( \div(\mathbf u) q - p \div(\mathbf v) \right)
+
    \int_{\partial\Omega} \left( p \mathbf n - 
 \frac{\partial \mathbf u}{\partial \mathbf n}\right)
     \cdot \mathbf v
=\int_\Omega \mathbf f \cdot \mathbf v 
\label{eq:tuto-stk-varform}
\end{equation}
%
Let us suppose now that $(\mathbf v,q) \in [H_0^1(\Omega)]^d \times L_0^2(\Omega)$, the variationnal formulation leads to:
Find $(\mathbf{u},p)\in [H_g^1(\Omega)]^d\times L_0^2(\Omega)$ such that for all $(\mathbf{v},q) \in [H_0^1(\Omega)]^d \times L_0^2(\Omega)$
\begin{equation}
\int_\Omega \mu \nabla \mathbf{u} :\nabla \mathbf{v}
+\int_\Omega \left( \nabla\cdot\mathbf{u} q - p \nabla\cdot\mathbf{v} \right)
=\int_\Omega \mathbf{f} \cdot \mathbf{v}
\end{equation}
Or equivalently:
\begin{equation}
  a((\mathbf{u},p),(\mathbf{v},q)) = l((\mathbf{v},q))
\end{equation}
where $a$ is a bilinear form, continuous, coercive and where $l$ is a linear form.
Let's see the \feel code corresponding to this mathematical statement.
(source \textcolor{magenta}{"doc/manual/tutorial/mystokes.cpp"}).
We suppose for this example the viscosity $\mu=1$ and $\mathbf f = 0$.
%
\vspace{2mm}
\lstinputlisting[linerange=marker_main-endmarker_main]{tutorial/mystokes.cpp}
\vspace{2mm}
%

The procedure to create the mesh is very simple.
You have to provide to the command line (or via the cfg file) the gmsh.filename option.
You can provide a geo or a msh file (created via gmsh).

As for the laplacian problem, the code is very closed to the mathematical formulation.
We define the product of function spaces for the flow speed and the flow pressure
using \lstinline!THch<order>()~\footnote{TH stands for Taylor-Hoods;}! function which is \lstinline!Pch<N+1>!$\times$ \lstinline!Pch<N>!
for respectively flow speed and pressure spaces.
We take an element 
$U=\left(
    \begin{array}{c}
        u \\
        p \\
    \end{array}
\right)
$
in this space. Then we define the integrals of the variationnal formulation
for the left and the right hand side. Finally, we apply the Poiseuille profile on the boundary.
We call the solver to resolve the problem (\ref{eq:tuto-stk-2}).



