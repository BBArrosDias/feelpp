\feelchapter{2D Maxwell simulation in a diode}
            {2D Maxwell simulation in a diode}
            {Thomas Strub, Philippe Helluy, Christophe Prud'homme}
            {cha:maxwell-2d}

\section{Description}
\label{sec:description}

The Maxwell equations read:


\begin{eqnarray*}
\frac{-1}{c^{2}}\frac{\delta}{\delta t}E+\nabla\times B & = & \mu_{0}J\\
B_{t}+\nabla\times E & = & 0\\
\nabla.B & = & 0\\
\nabla.E & = & \frac{\rho}{\epsilon_{o}}
\end{eqnarray*}


where $E$ représente le champ electrique, $B$ le champ magnétique,
$J$ la densité de courant, $c$ la célérité de la lumière, $\rho$
la densité de charge électrique, $\mu_{0}$ la perméabilité du vide
et $\epsilon_{0}$ la permittivité du vide.

Dans le milieu industrielle, notament en aéronautique, les systèmes
produits doivent vérifier certaines normes par exemple la réception
d'une onde electromagnétique émise par un radar ne doit pas entrainer
l'inefficassité d'une partie ou de tout le matériel présent dans le
système.

Ainsi, la simulation de tels situations permet lors du développement
ou lors de la certification d'un nouveau produit de tester sa réaction
à de tels agressions.

Remarquons également que les deux dernières équations sont en fait
des conditions initiales, en effet si on les suppose vraies à l'instant
$t=0$ alors elle peuvent êtres deduite des deux premières.

Supposons qu'à $t=0$, on a
\begin{eqnarray*}
\nabla.B & = & 0\\
\nabla.E & = & \frac{\rho}{\epsilon_{o}}
\end{eqnarray*}


i.e.\begin{eqnarray*}
\frac{\delta B_{x}}{\delta x}(t=0)+\frac{\delta B_{y}}{\delta y}(t=0)+\frac{\delta B_{z}}{\delta z} & (t=0)= & 0\\
\frac{\delta E_{x}}{\delta x}(t=0)+\frac{\delta E_{y}}{\delta y}(t=0)+\frac{\delta E_{z}}{\delta z} & (t=0)= & \frac{\rho}{\epsilon_{o}}\end{eqnarray*}


En dérivant la première de ces deux équations par rapport au temps,
on obtient :

\begin{eqnarray*}
\frac{\delta}{\delta t}\frac{\delta}{\delta x}B_{x}+\frac{\delta}{\delta t}\frac{\delta}{\delta y}B_{y}+\frac{\delta}{\delta t}\frac{\delta}{\delta z}B_{z} & = & \frac{\delta}{\delta x}\left(\frac{\delta}{\delta y}E_{z}-\frac{\delta}{\delta z}E_{y}\right)+\frac{\delta}{\delta y}\left(\frac{\delta}{\delta z}E_{x}-\frac{\delta}{\delta x}E_{z}\right)+\frac{\delta}{\delta z}\left(\frac{\delta}{\delta x}E_{y}-\frac{\delta}{\delta y}E_{x}\right)\\
 & = & 0\end{eqnarray*}


puisque

\begin{equation}
  \label{eq:3}
  B_{t}+\nabla\times E=0
\end{equation}



Donc, pour tout $t\geq0$,
\begin{equation}
  \label{eq:4}
  \nabla.B(t)=\nabla.B(0)=0
\end{equation}

On déduit de la meme manière la deuxième équation, en utilisant l'équation de
conservation de la charge :

\begin{equation}
  \label{eq:2}
  \frac{\delta}{\delta t}\rho+\Delta.\; J=0
\end{equation}



\section{Variational formulation}
\label{sec:vari-form}

\section{Implementation}
\label{sec:implementation}

\section{Numerical Results}
\label{sec:numerical-results}


%%% Local Variables:
%%% coding: utf-8
%%% mode: latex
%%% TeX-PDF-mode: t
%%% TeX-parse-self: t
%%% x-symbol-8bits: nil
%%% TeX-auto-regexp-list: TeX-auto-full-regexp-list
%%% TeX-master: "../feel-manual"
%%% ispell-local-dictionary: "american"
%%% End:
