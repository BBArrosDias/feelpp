\feelchapter{Non-Linear examples}
            {Non-Linear examples}
            {Christophe Prud'homme}
            {cha:non-linear-ex}

\section{Solving nonlinear equations}
\label{sec:nonlinear-equations}

\feel allows to solve nonlinear equations thanks to its interface to
the interface to the PETSc nonlinear solver library. It requires the
implementation of two extra functions in your application that will
update the jacobian matrix associated to the tangent problem and the
residual.

Consider that you have an application class \lstinline!MyApp! with a
backend as data member
\begin{lstlisting}{gobble=2}
#include <feel/feelcore/feel.hpp>
#include <feel/feelcore/application.hpp>
#include <feel/feelalg/backend.hpp>
namespace Feel {

class MyApp : public Application
{
  public:

  typedef Backend<double> backend_type;
  typedef boost::shared_ptr<backend_type> backend_ptrtype;

  MyApp( int argc, char** argv,
  AboutData const& ad, po::options_description const& od )
  :
  // init the parent class
  Application( argc, argv, ad, od ),
  // init the backend
  M_backend( backend_type::build( this->vm() ) ),
  {
    // define the callback functions (works only for the PETSc backend)
    M_backend->nlSolver()->residual =
      boost::bind( &self_type::updateResidual, boost::ref( *this ), _1, _2 );
    M_backend->nlSolver()->jacobian =
      boost::bind( &self_type::updateJacobian, boost::ref( *this ), _1, _2 );

  }
  void updateResidual( const vector_ptrtype& X, vector_ptrtype& R )
  {
    // update the matrix J (Jacobian matrix) associated
    // with the tangent problem
  }
  void updateJacobian( const vector_ptrtype& X, sparse_matrix_ptrtype& J)
  {
    // update the vector R associated with the residual
  }
  void run()
  {

    //define space
    Xh...
    element_type u(Xh);
    // initial guess is 0
    u = project( M_Xh, elements(mesh), constant(0.) );
    vector_ptrtype U( M_backend->newVector( u.functionSpace() ) );
    *U = u;

    // define R and J
    vector_ptrtype R( M_backend->newVector( u.functionSpace() ) );
    sparse_matrix_ptrtype J;

    // update R
    updateJacobian( U, R );
    // update J
    updateResidual( U, J );

    // solve using non linear methods (newton)
    // tolerance : 1e-10
    // max number of iterations : 10
    M_backend->nlSolve( J, U, R, 1e-10, 10 );

    // the soluution was stored in U
    u = *U;
  }
  private:

  backend_ptrtype M_backend;
};
} // namespace Feel
\end{lstlisting}

The function \lstinline!updateJacobian! and \lstinline!updateResidual!
implement the assmebly of the matrix $J$ (jacobian matrix) and the
vector $R$ (residual vector) respectively.

\subsection{A first nonlinear problem}
\label{sec:bratu}

As a simple example, let $\Omega$ be a subset of $\mathbb{R}^d, d=1,2,3$,
(\emph{i.e.} $\Omega=[-1,1]^d$) with boundary $\partial
\Omega$. Consider now the following equation and boundary condition
\begin{equation}
  \label{eq:29}
  -\Delta u + u^\lambda = f,\quad u = 0 \text{ on } \partial \Omega.
\end{equation}
where $\lambda \in \mathbb{R_+}$ is a given parameter and $f=1$.


\begin{nota}
  To be described in this section. For now see
  \texttt{doc/tutorial/nonlinearpow.cpp} for an implementation of this
  problem.
\end{nota}

\subsection{Simplified combustion problem: Bratu}
\label{sec:bratu}

As a simple example, let $\Omega$ be a subset of $\mathbb{R}^d, d=1,2,3$,
(\emph{i.e.} $\Omega=[-1,1]^d$) with boundary $\partial
\Omega$. Consider now the following equation and boundary condition
\begin{equation}
  \label{eq:29}
  -\Delta u + \lambda e^u = f,\quad u = 0 \text{ on } \partial \Omega
\end{equation}
where $\lambda$ is a given parameter. Ceci est généralement appellé le
problème de Bratu et apparaît lors de la simplification de modèles de
processus de diffusion non-linéaires par exemple dans le domaine de la
combustion.

\begin{nota}
  To be described in this section. For now see
  \texttt{doc/tutorial/bratu.cpp} for an implementation of this
  problem.
\end{nota}

%%% Local Variables:
%%% coding: utf-8
%%% mode: latex
%%% TeX-PDF-mode: t
%%% TeX-parse-self: t
%%% x-symbol-8bits: nil
%%% TeX-auto-regexp-list: TeX-auto-full-regexp-list
%%% TeX-master: "../feel-manual"
%%% ispell-local-dictionary: "american"
%%% End:

