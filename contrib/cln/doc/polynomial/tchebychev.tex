%% This LaTeX-file was created by <bruno> Sun Feb 16 14:05:43 1997
%% LyX 0.10 (C) 1995 1996 by Matthias Ettrich and the LyX Team

%% Don't edit this file unless you are sure what you are doing.
\documentclass[12pt,a4paper,oneside,onecolumn]{article}
\usepackage[]{fontenc}
\usepackage[latin1]{inputenc}
\usepackage[dvips]{epsfig}

%%
%% BEGIN The lyx specific LaTeX commands.
%%

\makeatletter
\def\LyX{L\kern-.1667em\lower.25em\hbox{Y}\kern-.125emX\spacefactor1000}%
\newcommand{\lyxtitle}[1] {\thispagestyle{empty}
\global\@topnum\z@
\section*{\LARGE \centering \sffamily \bfseries \protect#1 }
}
\newcommand{\lyxline}[1]{
{#1 \vspace{1ex} \hrule width \columnwidth \vspace{1ex}}
}
\newenvironment{lyxsectionbibliography}
{
\section*{\refname}
\@mkboth{\uppercase{\refname}}{\uppercase{\refname}}
\begin{list}{}{
\itemindent-\leftmargin
\labelsep 0pt
\renewcommand{\makelabel}{}
}
}
{\end{list}}
\newenvironment{lyxchapterbibliography}
{
\chapter*{\bibname}
\@mkboth{\uppercase{\bibname}}{\uppercase{\bibname}}
\begin{list}{}{
\itemindent-\leftmargin
\labelsep 0pt
\renewcommand{\makelabel}{}
}
}
{\end{list}}
\def\lxq{"}
\newenvironment{lyxcode}
{\list{}{
\rightmargin\leftmargin
\raggedright
\itemsep 0pt
\parsep 0pt
\ttfamily
}%
\item[]
}
{\endlist}
\newcommand{\lyxlabel}[1]{#1 \hfill}
\newenvironment{lyxlist}[1]
{\begin{list}{}
{\settowidth{\labelwidth}{#1}
\setlength{\leftmargin}{\labelwidth}
\addtolength{\leftmargin}{\labelsep}
\renewcommand{\makelabel}{\lyxlabel}}}
{\end{list}}
\newcommand{\lyxletterstyle}{
\setlength\parskip{0.7em}
\setlength\parindent{0pt}
}
\newcommand{\lyxaddress}[1]{
\par {\raggedright #1 
\vspace{1.4em}
\noindent\par}
}
\newcommand{\lyxrightaddress}[1]{
\par {\raggedleft \begin{tabular}{l}\ignorespaces
#1
\end{tabular}
\vspace{1.4em}
\par}
}
\newcommand{\lyxformula}[1]{
\begin{eqnarray*}
#1
\end{eqnarray*}
}
\newcommand{\lyxnumberedformula}[1]{
\begin{eqnarray}
#1
\end{eqnarray}
}
\makeatother

%%
%% END The lyx specific LaTeX commands.
%%

\pagestyle{plain}
\setcounter{secnumdepth}{3}
\setcounter{tocdepth}{3}
\begin{document}

The Tschebychev polynomials (of the 1st kind)  \( T_{n}(x) \) are defined through
the recurrence relation


\[
T_{0}(x)=1\]



\[
T_{1}(x)=x\]



\[
T_{n+2}(x)=2x\cdot T_{n+1}(x)-T_{n}(x)\]
 for  \( n\geq 0 \).

\begin{description}

\item [Theorem:]~

\end{description}

 \( T_{n}(x) \) satisfies the differential equation  \( (x^{2}-1)\cdot T_{n}^{''}(x)+x\cdot T_{n}^{'}(x)-n^{2}\cdot T_{n}(x)=0 \) for all  \( n\geq 0 \).

\begin{description}

\item [Proof:]~

\end{description}

Let  \( F:=\sum ^{\infty }_{n=0}T_{n}(x)z^{n} \) be the generating function of the sequence of polynomials. The
recurrence is equivalent to the equation 
\[
(1-2x\cdot z+z^{2})\cdot F=1-x\cdot z\]


\begin{description}

\item [Proof~1:]~

\end{description}

 \( F \) is a rational function in  \( z \),  \( F=\frac{1-xz}{1-2xz+z^{2}} \). From the theory of recursions with
constant coefficients, we know that we have to perform a partial fraction
decomposition. So let  \( p(z)=z^{2}-2x\cdot z+1 \) be the denominator and  \( \alpha =x+\sqrt{x^{2}-1} \) and  \( \alpha ^{-1} \) its zeroes.
The partial fraction decomposition reads 
\[
F=\frac{1-xz}{1-2xz+z^{2}}=\frac{1}{2}\left( \frac{1}{1-\alpha z}+\frac{1}{1-\alpha ^{-1}z}\right) \]
 hence  \( T_{n}(x)=\frac{1}{2}(\alpha ^{n}+\alpha ^{-n}) \). Note that the
field  \( Q(x)(\alpha ) \), being a finite dimensional extension field of  \( Q(x) \) in characteristic
0, has a unique derivation extending  \( \frac{d}{dx} \) on  \( Q(x) \). We can therefore try
to construct an annihilating differential operator for  \( T_{n}(x) \) by combination
of annihilating differential operators for  \( \alpha ^{n} \) and  \( \alpha ^{-n} \). In fact,  \( L_{1}:=(\alpha -x)\frac{d}{dx}-n \) satisfies
 \( L_{1}[\alpha ^{n}]=0 \), and  \( L_{2}:=(\alpha -x)\frac{d}{dx}+n \) satisfies  \( L_{2}[\alpha ^{-n}]=0 \). A common multiple of  \( L_{1} \) and  \( L_{2} \) is easily found
by solving an appropriate system of linear equations:

 \( L=(x^{2}-1)\left( \frac{d}{dx}\right) ^{2}+x\frac{d}{dx}-n^{2}=\left( (\alpha -x)\frac{d}{dx}+n\right) \cdot L_{1}=\left( (\alpha -x)\frac{d}{dx}-n\right) \cdot L_{2} \)

It follows that both  \( L[\alpha ^{n}]=0 \) and  \( L[\alpha ^{-n}]=0 \), hence  \( L[T_{n}(x)]=0 \).

\begin{description}

\item [Proof~2:]~

\end{description}

Starting from the above equation, we compute a linear relation for
the partial derivatives of  \( F \). Write  \( \partial _{x}=\frac{d}{dx} \) and  \( \Delta _{z}=z\frac{d}{dz} \). One computes


\[
\left( 1-2xz+z^{2}\right) \cdot F=1-xz\]

\[
\left( 1-2xz+z^{2}\right) ^{2}\cdot \partial _{x}F=z-z^{3}\]

\[
\left( 1-2xz+z^{2}\right) ^{3}\cdot \partial _{x}^{2}F=4z^{2}-4z^{4}\]

\[
\left( 1-2xz+z^{2}\right) ^{2}\cdot \Delta _{z}F=xz-2z^{2}+xz^{3}\]

\[
\left( 1-2xz+z^{2}\right) ^{3}\cdot \partial _{x}\Delta _{z}F=z+2xz^{2}-6z^{3}+2xz^{4}+z^{5}\]

\[
\left( 1-2xz+z^{2}\right) ^{3}\cdot \Delta _{z}^{2}F=xz+(2x^{2}-4)z^{2}-(2x^{2}-4)z^{4}-xz^{5}\]


Solve a  \( 6\times 6 \) system of linear equations over  \( Q(x) \) to get 
\[
x\cdot \partial _{x}F+(x^{2}-1)\cdot \partial _{x}^{2}F-\Delta _{z}^{2}F=0\]


This is equivalent to the claimed equation  \( (x^{2}-1)\cdot T_{n}^{''}(x)+x\cdot T_{n}^{'}(x)-n^{2}\cdot T_{n}(x)=0 \).

\begin{lyxsectionbibliography}

\item [1] Bruno Haible: D-finite power series in several variables. \em Diploma
thesis, University of Karlsruhe, June 1989. \em Sections 2.12 and
2.15.

\end{lyxsectionbibliography}

\end{document}
